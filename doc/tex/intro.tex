% =============================================================================

\paragraph{Implementation attacks}

Among documents associated with the so-called Snowden revelations, the NSA 
exploit catalogue is arguably the most concrete and illuminating.  Analysis
of the content has led many commentators to remark on a lack of traditional
cryptanalytic breakthroughs.  Put simply, and although interpreting various
aspects at face value, the implication is that modern cryptography indeed
offers the security margins expected.  The state of end-point security is, 
however, {\em significantly} less positive in that similar security margins
are either degraded or rendered moot by deficiencies in the implementation 
of hardware, software, or a combination of both.  The increased emphasis on
end-to-end security, while sane, makes this particularly problematic: the
end-points, where security-critical storage and computation are paramount,
also seem demonstrably {\em in}secure.

A recurring exemplar is the potential for implementation attacks against a
given target; the mechanism used to mount such an attack will intrinsically
depend on the concrete implementation, and hence behaviour of said target,
vs. an abstract design for example.
The importance of implementation attacks can be motivated by observing that
an given attacker might not behave as they {\em should} (i.e., as expected)
but however they {\em can}.  To render ineffective the security afforded by
some encryption scheme, for example, they might target
a) the concrete, in practice implementation
   {\em or}
b) the abstract, on paper    design
of said scheme.  Typically, they will select the option for which they can
succeed most easily (colloquially, the weakest link).  If security of the
scheme is (provably) related to some hard Mathematical problem, traditional
cryptanalysis is unlikely to be easy at all.  Opting for an implementation
attack, however, offers the potential to bypass the associated difficulty:
the hard problem might be rendered moot if the attack instead focuses on a
feature (e.g., flaw) in the implementation of it.

% -----------------------------------------------------------------------------

\paragraph{Side-channel   attacks}

At a high level, implementation attacks fall into two sub-classes:

\begin{enumerate}
\item An active  attack
      is st. the attacker influences behaviour of the target;
      they leverage direct inputs and outputs plus (optionally) indirect  input,
      and are typically described as 
      fault
      attacks.
\item A  passive attack
      is st. the attacker   observes behaviour of the target;
      they leverage direct inputs and outputs plus (optionally) indirect output,
      and are typically described as 
      side-channel (or information leakage)
      attacks.
\end{enumerate}

\noindent
In the case of passive attacks, a given indirect output is referred to as
a (or the) side-channel; said channel will act as a mechanism though which
information leaks.  Although variants can make sense, typical side-channel
attacks will involve two phases in some form:
a) an acquisition    phase, 
   where interaction with the attack target yields an associated data set
   through observation of associated behaviour,
   then
b) a (crypt)analysis phase,
   where data set is used satisfy a goal such as 
    (generic) leakage assessment
   or 
   (specific) key recovery.
As a motivating example, consider a power-based side-channel attack against
a CMOS-based attack target.  Each interaction with the attack target would
yield a trace of power consumption, i.e., a sequence of samples
\[
T = \LIST{ T_0, T_1, \ldots, T_{n-1} } ,
\]
which will be
{\em design}-dependent
(e.g., dependent on the number, type, and connection between transistors)
plus
{\em  data}-dependent
(i.e., change based on switching activity of said transistors);
the acquisition phase yields a set of such traces, each of which includes
a signal (i.e., data of interest) plus noise, while the analysis phase is
tasked with using that data set to satisfy the attack goal.

Evolution of the state-of-the-art wrt. such phases are often described as
symbiotic, and clearly {\em both} are integral to a concrete attack.  For
example, an acquired data set with no means to cryptanalytically exploit
it is somewhat redundant; likewise, improvement to the acquisition phase
may be able to support more effective cryptanalysis (e.g., as a result of
higher SNR).

% -----------------------------------------------------------------------------

\paragraph{Approaches to mitigation}

In addition to the relative ineffectiveness of cryptanalysis against modern
cryptographic designs, a variety of trends illustrate why concrete attacks
based on these concepts are such a threat; a non-exhaustive set of examples
follows.

\begin{enumerate}

\item There has been an obvious societal shift toward demand for, and even
      dependency on digital information systems; such systems often process
      security- and identity-related data, while also operating within an
      adversarial environment.

\item There is often significant potential for application at scale, which
      typically amplifies the potency of attacks.  Examples include where
      a) increased connectivity permits attacks to be mounted remotely via
         a network (e.g., targeting a web-server),
         or
      b) a large number of similarly viable targets exist (e.g., massive
         deployment of some embedded devices such as a smart-phone, or IoT
         device).

\item In   theory,
      increased complexity   of      such targets
      offers an intuitive deterrent against attack;
      contact-based side-channels related to power consumption {\em may} be
      rendered infeasible by electronic or architectural complexity.
      In practice,
      however, the ``attacks only get better'' principle is evident: use of
      contact-less side-channels, such as EM, have been harnessed against
      System-on-Chip (SoC) components, 
      smart-phones,
      and even
      full-scale personal computers
      operating at multi-gigahertz clock frequencies.

\end{enumerate}

% TODO

\noindent
As a result, it has been suggested that treating side-channels security as a
first-class goal is
a) {\em necessary},
   and thus
b) the {\em duty} of hardware security Engineers.

Along such lines, robust mitigation of side-channel attacks is, on one hand,
supported by a rich, mature corpus of literature; mitigation techniques (or
countermeasures) are increasingly well understood.  On the other hand, the
concrete vs. abstract nature of side-channel attacks fundamentally dictates
challenges wrt. selecting, instrumenting, and assessing the efficacy of such
mitigations.


%Understanding attacks and associated mitigations demands 
%
%Particularly where the side-channel in question is
%analogue (e.g., power consumption)
%vs.
% digital (e.g., execution time)
%in nature, the first phase constitutes a challenge that can lli
%
%
%Examples include
%capital investment wrt. equipment, maintenance and operational experience, and logistics;
%
%even though low-cost platforms such as
%[ChipWhisperer](https://newae.com/tools/chipwhisperer)
%help to some extent, the concept of
%[Total Cost of Ownership (TCO)](https://en.wikipedia.org/wiki/Total_cost_of_ownership)
%means they can still impact on how applicable the approach is.

 minimise total cost of ownership
and hence maximise return on investment.

% -----------------------------------------------------------------------------

\paragraph{\LABID: side-channel analysis ``as a service''}

% TODO

% =============================================================================
