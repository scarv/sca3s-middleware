% =============================================================================

\paragraph{xxx}

Among documents associated with the so-called Snowden revelations, the NSA 
exploit catalogue is arguably the most concrete and illuminating.  Analysis
of the content has led many commentators to remark on a lack of traditional
cryptanalytic breakthroughs.  Put simply, and although interpreting various
aspects at face value, the implication is that modern cryptography indeed
offers the security margins expected.  The state of end-point security is, 
however, {\em significantly} less positive in that similar security margins
are either degraded or rendered moot by deficiencies in the implementation 
of hardware, software, or a combination of both.  The increased emphasis on
end-to-end security, while sane, makes this particularly problematic: the
end-points, where security-critical storage and computation are paramount,
also seem demonstrably {\em in}secure.

A recurring exemplar is the potential for implementation attacks against a
given target; the mechanism used to mount such an attack will intrinsically
depend on the concrete implementation, and hence behaviour of said target,
vs. an abstract specification for example.
The importance of implementation attacks can be motivated by observing that
an given attacker might not behave as they {\em should} (i.e., as expected)
but however they {\em can}.  To render ineffective the security afforded by
some encryption scheme, for example, they might target
a) the concrete, in practice implementation
   {\em or}
b) the abstract, on paper    specification
of said scheme.  Typically, they will select the option for which they can
succeed most easily (colloquially, the weakest link).  If security of the
scheme is (provably) related to some hard Mathematical problem, traditional
cryptanalysis is unlikely to be easy at all.  Opting for an implementation
attack, however, offers the potential to bypass the associated difficulty:
the hard problem might be rendered moot if the attack instead focuses on a
feature (e.g., flaw) in the implementation of it.

% -----------------------------------------------------------------------------

\paragraph{xxx}

Over the last $5$ years, (at least) two technology trends have acted to
exacerbate the resulting threat, and hence importance of this field.
First, the potential for application at scale will typically amplify the 
threat of {\em any} attack type.  Contemporary scenarios include when the 
attack is mounted remotely via a network (e.g., targeting a web-server), 
{\em or} where a large number of similarly viable targets exist (e.g., 
massive deployment of some embedded devices such as a smart-phone, or IoT
device).  
Second, the ability to consider traditional (especially EM-based) methods
of acquiring exploitable side-channel leakage has improved significantly
over the last few years (per the ``attacks only get better'' principle).  
Among selected examples, the last $2$ years have seen attacks on
System-on-Chip (SoC) components~\cite{complex1}, 
smart-phones~\cite{complex2},
and even
full-scale personal computers~\cite{complex3} 
operating at multi-gigahertz clock frequencies.  Again this is important,
because it further widens the attack surface: it could now be possible to
harness micro-architectural leakage via forms of acquisition more aligned 
with embedded targets (e.g., EM).

% -----------------------------------------------------------------------------

\paragraph{xxx}

However, at a high level, a given attack demands (at least) two phases:
a) acquisition of a data set,
   then
b) (crypt)analysis of said data set, which aims to satisfy a goal such as
   key recovery.
Particularly where the side-channel in question is
*analogue* (e.g., power consumption)
vs.
 *digital* (e.g., execution time)
in nature, the acquisition phase can present some significant
practical challenges.  Examples include
capital investment wrt. equipment, maintenance and operational experience, and logistics;
even though low-cost platforms such as
[ChipWhisperer](https://newae.com/tools/chipwhisperer)
help to some extent, the concept of
[Total Cost of Ownership (TCO)](https://en.wikipedia.org/wiki/Total_cost_of_ownership)
means they can still impact on how applicable the approach is.

% =============================================================================
