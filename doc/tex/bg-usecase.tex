% =============================================================================

\subsection{Potential use-cases}

a remote acquisition process is required, or at least preferred


a) bespoke or niche equipment is used, which then cannot be scaled to 
   match either the location and/or number of users,
   or
b) scientific tenets such as open access and reproducibility demand
   the same equipment be used 

% -----------------------------------------------------------------------------

\subsubsection{}
a side-channel oriented form of Continuous Integration (CI), potentially allowing small development teams to improve the resilience of their cryptographic implementations

% -----------------------------------------------------------------------------

\subsubsection{}
cryptographic design or standardisation process requires a forms of build-it, break-it, fix-it approach

%Whether selecting new or codifying existing technologies, the development
%of cryptographic standards is a fundamentally important challenge.  Only 
%somewhat orthogonal to the development of {\em written} standards, is the 
%development of de facto standard {\em implementations}: these software 
%components form concrete, ubiquitous realisations of theory that fill 
%critical, long-lived roles.
%In both cases, open contests have emerged as an important approach (or a
%component thereof); by allowing open analysis and debate, they typically 
%enhance the quality of and trust in the eventual outcome.  Initiated in
%$1997$ and concluding in $2000$, the AES process~\cite{aescontest} run by
%NIST is an obvious exemplar.  In fact, it encompassed elements of both
%cases above:
%not only was ``hardware and software suitability'' an explicit evaluation
%criteria for the new block cipher design, but the process begat various de 
%facto implementations that are still in wide usage.  To a lesser extent, a 
%similar situation is true of side-channel attacks.  For example, iterations 
%of the DPA contest~\cite{dpacontest} challenge participants to mount an
%efficient attack against a fixed target (using a set of pre-acquired 
%traces); this has predominantly been used compare and refine attack 
%techniques.

%Whether selecting new or codifying existing technologies, the development
%of cryptographic standards is an important challenge.  While traditional,
%written standards are one element of this challenge, the development of 
%de facto standard implementations are another: such software components 
%form concrete, ubiquitous realisations of theory that fill critical, and
%long-lived roles.
%
%Within this context, open contests have emerged as an important approach 
%(or component thereof); allowing open analysis and debate, they typically 
%enhance the quality of and trust in the eventual outcome.  Initiated in
%1997 and concluding in 2000, the AES process run by NIST is an exemplar
%in the sense it encompassed both elements outlined above: "hardware and 
%software suitability" was an stated evaluation criteria for new designs, 
%but the process also begat de facto implementations that are still now in
%wide use.  
%
%To some extent, the same argument is true in more applied fields such as
%that of side-channel attacks [1] (a branch of cryptographic engineering).
%For example, iterations of the DPA contest [2] challenge participants to 
%mount an efficient attack against a target implementation (based on use
%of pre-acquired data sets).  However, the open contest model implemented 
%by the DPA contest exhibits some arguable disadvantages.  In particular, 
%it is
%
%a) somewhat uni-directional, in the sense it has predominantly been used 
%   as a resource for comparing and refining attack techniques rather than
%   secure implementations, and
%b) somewhat non-continuous: there is typically a contest outcome, with no
%   interaction or iteration possible.
%
%We posit that a build-it, break-it, fix-it [4] style contest model, if it
%were possible, would resolve such disadvantages: the problem, and hence 
%crux of this proposal, is the lack of a suitable platform to operate such 
%a model within this context.

% -----------------------------------------------------------------------------

\subsubsection{}
educational infrastructure, e.g., forms of CTF-like contest

% =============================================================================
