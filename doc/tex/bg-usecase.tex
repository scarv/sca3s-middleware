% =============================================================================

\subsection{Potential use-cases}

% -----------------------------------------------------------------------------

\subsubsection{A mechanism to abstract the acquisition process}

The two processes constituting a given side-channel attack {\em could},
superficially, be described as
a)  lower-level, with more Engineering-oriented challenges,
   and
b) higher-level, with more Mathematics-oriented challenges,
respectively.  
In reality, the processes evolve and are used in a symbiotic manner: the 
analysis process should cater for any characteristics (e.g., SNR) of the 
associated acquisition process, for example, but could act as motivation
for improvements of it.  
That said, however, a subset of practitioners legitimately focus on the
latter alone.

% scientific tenets such as open access and reproducibility demand
%  the same equipment be used 

% -----------------------------------------------------------------------------

\subsubsection{A mechanism to manage logistical constraints}

Managing shared access to (potentially bespoke or niche) equipment can be 
difficult, especially so if issues of scale (e.g., the physical proximity, 
and number of users) are evident.  As such, \LABID {\em could} be viewed 
simply as a framework that addresses the associated logistical challenge.

Beyond this, however, various opportunities emerge from the remote use of
shared equipment via \LABID.  The concept of Continuous Integration (CI)
is one example.  Traditional CI is a development best-practice, the core 
tenet under which is for source code 
a) to be regularly (e.g., several times a day) integrated into a central repository 
   and, in doing so, 
b) subjected to a suite of automated analysis processes, e.g., to verify completion of the build process.
A central advantage of this approach is the early identification, and so
resolution, of functional defects.  
It is obviously attractive to consider security-focused instances of CI:
products such as Veracode Greenlight\footnote{
\URL{https://www.veracode.com/products/greenlight}
} deliver this by focusing on security-specific defects in software.  A
(small) step further would be to consider CI as applied to security from
a side-channel perspective.  In essence, \LABID offers a platform where a 
cryptographic implementation can, within an existing development workflow,
be automatically analysed to identify side-channel related defects (e.g., 
an above threshold level of generic leakage).
We posit that offering this capability, particularly to small development
teams, is an easy route to (incrementally) improving the global quality 
of cryptographic implementations.

% -----------------------------------------------------------------------------

\subsubsection{A mechanism to support design or standardisation}

Whether selecting new or codifying existing technologies, the development
of cryptographic standards is a fundamentally important challenge.  Only 
somewhat orthogonal to the development of {\em written} standards, is the 
development of de facto standard {\em implementations}: these software 
components form concrete, ubiquitous realisations of theory that fill 
critical, long-lived roles.
In both cases, open contests have emerged as an important approach (or a
component thereof); by allowing open analysis and debate, they typically 
enhance the quality of and trust in the eventual outcome.  Initiated in
$1997$ and concluding in $2000$, the AES process~\cite{SCARV:Burr:03},
led by NIST, is an obvious exemplar.  In fact, it encompassed elements of 
{\em both} arguments above: ``hardware and software suitability'' was an 
explicit evaluation criteria for the new block cipher design, plus the 
process begat various de facto implementations that are still in use.

To some extent, the same argument is true in more applied fields such as
that of side-channel attacks.  For example, each iteration of the DPA 
contest~\cite{SCARV:CDDEGGHKLLNOSSSSVWW:14} challenges participants to 
mount an efficient attack against a target implementation based on use
of pre-acquired data sets.  However, the open contest model implemented 
by the DPA contest exhibits some arguable disadvantages.  In particular, 
it is

\begin{itemize}
\item somewhat uni-directional, in the sense it has predominantly been 
      used as a resource for comparing and refining attack techniques 
      rather than secure implementations, 
      and
\item somewhat non-continuous: there is typically a contest outcome, 
      with no interaction or iteration possible thereafter.
\end{itemize}

\noindent
We posit that a build-it, break-it, fix-it~\cite{SCARV:RHPLMM:16} style 
contest model, if it were possible, could address such disadvantages;
\LABID plausibly represents a mechanism that could support such a model.

%@article{SCARV:CDDEGGHKLLNOSSSSVWW:14,
%  author  = {C. Clavier and J.-L. Danger and G. Duc and M.A. Elaabid and B. G\'{e}rard and S. Guilley and A. Heuser and M. Kasper and Y. Li and V. Lomn\'{e} and D. Nakatsu and K. Ohta and K. Sakiyama and L. Sauvage and W. Schindler and M. St\"{o}ttinger and N. Veyrat-Charvillon and M. Walle and A. Wurcker},
%  title   = {Practical improvements of side-channel attacks on {AES}: feedback from the 2nd {DPA} contest},
%  journal = {Journal of Cryptographic Engineering},
%  volume  = {4},
%  number  = {4},
%  pages   = {259--274},
%  year    = {2014}
%}
%
%@article{SCARV:Burr:03,
%  author  = {W.E. Burr},
%  title   = {Selecting the {Advanced} {Encryption} {Standard}},
%  journal = {IEEE Security and Privacy},
%  volume  = {1},
%  number  = {2},
%  pages   = {43--52},
%  year    = {2003}
%}
%
%@inproceedings{SCARV:RHPLMM:16,
%  author    = {A. Ruef and M. Hicks and J. Parker and D. Levin and M.L. Mazurek and P. Mardziel},
%  title     = {{Build It}, {Break It}, {Fix It}: Contesting Secure Development},
%  booktitle = {CCS},
%  year      = {2016},
%  url       = {http://builditbreakit.org}
%}

% -----------------------------------------------------------------------------

\subsubsection{A mechanism to support educational challenges}

%educational infrastructure, e.g., forms of CTF-like contest

% =============================================================================
